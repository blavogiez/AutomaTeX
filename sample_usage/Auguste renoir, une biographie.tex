\documentclass[12pt]{article}
   \usepackage[utf8]{inputenc}
   \usepackage[T1]{fontenc}
   \usepackage{babel}
   \usepackage{amsmath}
   \usepackage{amsfonts}
   \usepackage{amssymb}
   \usepackage{graphicx}
   \usepackage[left=2cm, right=2cm, top=2cm, bottom=2cm]{geometry}

   \title{Biographie d'Auguste Renoir}
   \author{}
   \date{}

   \begin{document}

   Auguste Renoir (February 25, 1841 – December 3, 1919) est un artiste français reconnu comme l'un des principaux représentants de l'impressionnisme. Il a produit plus de 6 000 œuvres à travers son long et fructueux parcours artistique.

   Renoir est né à Limoges, dans le département de la Haute-Vienne, en France. Sa famille était active dans l'industrie de la porcelaine, mais il montra un intérêt précoce pour l'art et commença à fréquenter une académie locale. En 1862, Renoir s'installa à Paris, où il trouva un poste de décorateur en faïence au Mintard factory.

   Cependant, la passion d'Auguste pour l'art l'amena rapidement vers les ateliers des peintres impressionnistes. Il participa à ses premiers Salons des Refusés en 1863 et exposa régulièrement avec Monet, Sisley, Cézanne, Degas, Pissarro et autres artistes impressionnistes.

   Renoir s'est particulièrement intéressé aux sujets de genre, aux natures mortes et aux portraits. Il est connu pour sa maîtrise remarquable des couleurs, son sens du mouvement dynamique et ses représentations des femmes en tant que symboles de beauté et de féminité.

   Au cours de sa carrière, Renoir a reçu plusieurs prix et reconnaissance internationale. Il est devenu un membre fondateur de la Société des Artistes Français et un membre honoraire de la Royal Academy de Londres.

   Auguste Renoir a continué à peindre jusqu'à sa mort en 1919, à l'âge de 78 ans. Son œuvre a eu un impact profond sur l'histoire de l'art et continue d'être appréciée et exposée dans les musées du monde entier.

   \end{document}

