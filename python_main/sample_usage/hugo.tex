\documentclass[12pt,a4paper]{article}
\usepackage[utf8]{inputenc}
\usepackage[T1]{fontenc}
\usepackage{lmodern}
\usepackage[frenchb]{babel}
\usepackage{graphicx}
\title{Victor Hugo : un grand écrivain français}
\author{}
\date{}
\begin{document}
\maketitle

Le 26 février 1802, Victor-Marie Hugo naît à Besançon. Il est le fils de Louis-Baptiste Hugo et Sophie Trebuchet.

À l'âge de deux ans, sa famille déménage à Saône pour travailler dans une entreprise d'imprimerie. C'est là qu'il acquiert une passion pour la lecture et la poésie.

En 1822, il publie son premier recueil de poèmes : les Odes et les Ballades. Sa notoriété grandit avec l'édition des Contes fantastiques (1830-1831) et La Légende des Siècles (1852-1860).

Sa plus grande œuvre, Notre-Dame de Paris (1831), est un grand succès. C'est cette même année qu'il épouse Adèle Foucher.

En 1848, il prend part à la révolution française en devenant le maire de Paris et soutient la candidature de Louis-Napoléon Bonaparte. Cependant, lorsqu'il arrive au pouvoir sous le nom de Napoléon III, Hugo est exilé en Angleterre.

En 1870, après l'abdication de Napoléon III, il peut revenir en France et s'installe à Guernesey. C'est là qu'il achève son chef-d'œuvre : Les Misérables (1862).

Victor Hugo meurt le 22 mai 1885 à Villequier, dans la Seine-Maritime. Il est inhumé à Paris au Panthéon, où il repose avec les autres grands écrivains français.

Il est génial
\end{document}
