\documentclass{article}
\usepackage[utf8]{inputenc}
\usepackage[french]{babel}

\title{Auguste Renoir: Maître de l’Impressionnisme} 
\author{}
\date{\today}

\begin{document} 
 
\maketitle

Auguste Renoir est l’un des peintres les plus emblématiques du mouvement impressionniste français. Né en 1841 à Limoges, il se distingue très tôt par son sens aigu de la couleur et de la lumière. Contrairement à d'autres impressionnistes, Renoir ne rejette pas totalement les formes traditionnelles, mais cherche à les enrichir par des touches légères et vibrantes. 

Parmi ses œuvres les plus célèbres, on peut citer \emph{Le Déjeuner des canotiers}, \emph{Bal du moulin de la Galette} ou encore \emph{Jeunes filles au piano}. Ses toiles célèbrent souvent la joie de vivre, les scènes de la vie quotidienne et les figures féminines.

Vers la fin de sa vie, bien qu'affaibli par l'arthrite, Renoir continue de peindre avec une énergie remarquable, utilisant parfois des pinceaux attachés à ses poignets. Il laisse derrière lui une œuvre riche, chaleureuse et profondément humaine, qui continue d’inspirer les amateurs d’art du monde entier.

On se souvient de lui comme d'un artiste qui a su transmettre la joie de vivre et la vie quotidienne sur ses toiles, marquées par une énergie remarquable jusqu’à sa dernière œuvre.


\end{document}









