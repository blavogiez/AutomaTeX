\documentclass{mytex}
\usepackage{xcolor}
\usepackage{fontawesome5}
\usepackage{babel}
\usepackage{tcolorbox}
\usepackage{uml} % Ce package permet des diagrammes UML simples
\tcbuselibrary{skins, listings}
\newtcblisting{codebox}{
	enhanced, 
	colback=black!3!white, 
	toprule=1pt, 
	bottomrule=1pt,
	leftrule=0pt, 
	rightrule=0pt, 
	arc=0mm, 
	attach boxed title to top left={yshift=-9pt, xshift=4pt},
	title = Java,
	coltitle= black, 
	boxed title style={colback=white},
	listing only,
	listing options={
		language=java,
		breaklines,
		keywordstyle = \color{green!60!black},
		keywordstyle = {[2]\color{orange}},
		commentstyle = \color{gray}\itshape,
		stringstyle  = \color{blue},
		showstringspaces=false,
		numbers = left,
		numberstyle = \tiny,
		stepnumber=1,
		numbersep = 20pt
	},
}
% pour les accents
\lstset{
	literate=
	{é}{{\'e}}1
	{è}{{\`{e}}}1
	{ê}{{\^{e}}}1
	{ë}{{\¨{e}}}1
	{É}{{\'{E}}}1
	{Ê}{{\^{E}}}1
	{û}{{\^{u}}}1
	{ù}{{\`{u}}}1
	{â}{{\^{a}}}1
	{à}{{\`{a}}}1
	{Â}{{\^{A}}}1
	{ç}{{\c{c}}}1
	{Ç}{{\c{C}}}1
	{ô}{{\^{o}}}1
	{Ô}{{\^{O}}}1
	{î}{{\^{i}}}1
	{Î}{{\^{I}}}1,
}

\title{Rapport ECL - Template} %Titre du fichier

\begin{document}
	
	
	
%----------- Informations du rapport ---------

\titre{Portfolio} %Titre du fichier .pdf
\UE{Communication} %Nom de la UE
\sujet{Portfolio} %Nom du sujet

\enseignant{Cécile \textsc{De Coninck}}

\eleves{Baptiste \textsc{Lavogiez}}

%----------- Initialisation -------------------

\fairemarges %Afficher les marges
\fairepagedegarde %Créer la page de garde
\tabledematieres %Créer la table de matières

%------------ Corps du rapport ----------------

\section{Introduction}

\subsection{Contexte}

\tsec{Les SAé : c'est quoi ?}

Dans le monde du BUT, les SAé sont des \textit{Situation d'apprentissage et d'évaluation}.
En quelque sorte, ce sont des projets, énoncés avec un cahier des charges, que nous devons réaliser afin de développer des compétences pratiques.
Ce mode d'apprentissage permet de développer tôt dans les études une capacité d'adaptation aux situations courantes du monde du travail et de l'informatique, tout en travaillant dans une équipe.
Il faut y voir une émulation d'un projet d'entreprise.

Personnellement, j'y applique une rigueur professionnelle, autant pour m'habituer à ce mode de fonctionnement que pour présenter les projets les plus beaux possibles.
La conception et réalisation de projet sont des aspects de l'informatique que j'aime vraiment, et j'aime à faire des rendus impeccables et clairement expliqués.

\subsection{Le déroulement de ce portfolio}

Ce document, ou ce \textbf{Portfolio}, vise à présenter tous nos projets sous 3 dogmes :

\begin{itemize}
\item{La présentation}
\item{L'application}
\item{Ce que nous avons appris}
\end{itemize}

\tsec{Présentation}
Cette partie présente en quelques lignes les grandes lignes du projet ; l'énoncé, le développement et le rendu.

\tsec{Quelle était l'application ?}
Cette partie présente l'utilité claire du projet et sa réapplicabilité dans d'autres domaines.

\tsec{Qu'ai-je appris ?}
Cette partie présente les compétences acquises après la réalisation du projet.

\section{SAÉ 1.02 - Réalisation d'un jeu en iJava}

\subsection{Présentation}

Ce projet consiste en la réalisation d'un jeu en iJava, servant ainsi à démontrer nos capacités premières de conception et de réalisation d'un projet simple avec cahier des charges et rendu progressif.

Ce projet était initialement prévu pour être réalisé en binôme, mais il a été développé seul en raison de la réorientation de l'autre membre.

\subsection{Quelle était l'application ?}

\textit{Le jeu réalisé est donc Javazonia !}

Javazonia est un jeu qui allie culture et géographie. Le joueur doit répondre à des questions sur un pays aléatoire et le placer correctement sur une carte interactive. En cas de succès, il peut créer un nouveau pays en remplaçant l'ancien, mettant ainsi sa créativité au premier plan !

\begin{itemize}
	\item Plus qu'un simple quiz, le joueur doit placer le pays sur une carte.
	\item Les résultats sont enregistrés et un classement permet de voir les statistiques de tous les joueurs.
	\item Conçue pour être simple d'utilisation, quel que soit l'âge.
\end{itemize}

\subsection{En images}

\insererfigure{img/102/1.png}{4cm}{}{fig::}

\insererfigure{img/102/2.png}{10cm}{}{fig::}

\insererfigure{img/102/3.png}{7cm}{}{fig::}

\subsection{Qu'ai-je appris ?}

La rigueur et la capacité à faire comprendre notre code à d'autres personnes étaient les principaux points abordés : j'ai fait attention à commenter et documenter le fonctionnement de l'application au maximum

\tsecnonum{Compétences techniques}
\begin{itemize}
	\item Développement d'un jeu en Java
	\item Utilisation de cartes interactives
	\item Gestion des entrées utilisateur
\end{itemize}

\tsecnonum{Savoir-être}
\begin{itemize}
	\item Rigueur dans le commentaire et la documentation du code
	\item Capacité à travailler en autonomie
	\item Créativité dans la conception de fonctionnalités de jeu
\end{itemize}

\section{SAÉ 1.03 - Installation de machines virtuelles}

\subsection{Présentation}

Ce projet fait partie de la ressource Systèmes. Il a pour objectif de nous introduire à l'installation et à la configuration de machines virtuelles basées sur Linux (Debian 12). 

Cette SAé se faisait seul. 

Il n'y avait pas de rendu informatique, car l'évaluation était un contrôle TP. Néanmoins, un court manuel d'anglais a du être rédigé. Cette SAé est sûrement la plus mineure de toutes les SAé réalisées.

\subsection{Quelle était l'application ?}

L'utilité claire est de nous former à l'environnement Unix et aux machines virtuelles. Cela m'a permis par la suite d'utiliser couramment des machines virtuelles en allant plus loin que ce que j'ai pu faire avant la formation.

\subsection{Qu'ai-je appris ?}

J'ai donc appris à installer et configurer des machines virtuelles !

\tsecnonum{Compétences techniques}
\begin{itemize}
	\item Installation et configuration de machines virtuelles
	\item Utilisation de l'environnement Unix
\end{itemize}

\tsecnonum{Savoir-être}
\begin{itemize}
	\item Autonomie dans l'apprentissage de nouveaux outils
	\item Adaptabilité à de nouveaux environnements de travail
\end{itemize}

\section{SAÉ 1.04 - Base de données Access}

\subsection{Présentation}

L’objectif de ce projet est de concevoir et mettre en place un système fictif de base de données pour une entreprise tirée au sort (ici, une agence de voyage). Les logiciels utilisés sont PowerAMC et Microsoft Access.

Ce projet insistera plus sur le côté conceptuel de la base de données que sur de la demande de requêtes pure. Un aspect bien plus profond du SQL sera donné dans le projet du Semestre 2.

La base de données Access est entièrement accessible dans le dossier sous-jacent.

\subsection{Quelle était l'application ?}

L'utilité du projet est de concevoir nous-mêmes une base de données en nous mettant dans la peau d'un dirigeant d'entreprise. Réfléchir à un modèle, puis revenir sur un autre en se rendant compte d'une incohérence passée, était l'essence de ce projet et il nous a appris à réfléchir de façon beaucoup plus mature et efficace lorsque l'on conçoit un modèle. Après ce projet, il y a vraiment eu une différence entre ce qu'il était important de modéliser ou non.

\subsection{En images}

\insererfigure{img/104/1.png}{10cm}{}{fig::}

\insererfigure{img/104/2.png}{10cm}{}{fig::}

\insererfigure{img/104/3.png}{10cm}{}{fig::}

\subsection{Qu'ai-je appris ?}

J'ai appris à me mettre dans la peau de plusieurs personnes afin de réaliser ce projet :

\begin{itemize}
	\item Un dirigeant de start-up
	\item Un développeur de base de données
	\item Un utilisateur de cette base (un employé, typiquement)
\end{itemize}

En somme, j'ai vu l'importance de se mettre à la place de chaque acteur du logiciel que l'on conçoit afin qu'il soit développé d'une façon la plus orientée possible à chacun.

\tsecnonum{Compétences techniques}
\begin{itemize}
	\item Conception de bases de données avec PowerAMC
	\item Utilisation de Microsoft Access pour la gestion de bases de données
	\item Réflexion sur la modélisation des données
\end{itemize}

\tsecnonum{Savoir-être}
\begin{itemize}
	\item Capacité à se mettre dans la peau de différents acteurs d'un projet
	\item Réflexion mature et efficace sur la conception de modèles
\end{itemize}

\section{SAÉ 1.05-06 - Réalisation d'un site Web}

\subsection{Présentation}

Cette SAé est celle avec l'objectif le plus simple, et pourtant elle n'est pas la plus simple ; il s'agit ici de développer un site Web. 

En effet, tous les informaticiens sont passés par là. Cette fois-ci, en groupe de 3, nous devions nous conformer à un cahier des charges afin de réaliser un site comprenant un menu navigable, un formulaire d'inscription et un style dense.Nous avons donc élaboré des concepTS de menu navigable, formulaire d'inscription et style dense, chacun adapté aux besoins spécifiques du site.Nous avons élaboré des concepTS de menu navigable, formulaire d'inscription et style dense, chacun adapté aux besoins spécifiques du site.

Nous avons donc travaillé ensemble pour définir les compétences techniques et le savoir-être nécessaires à la ré realisation du site Web, en prenant en compte les spécifications du cahier des charges.

\subsection{Quelle était l'application ?}

L'utilité claire est de convertir une idée, un concept en un site Web. La réalisation d'un site est sûrement une des tâches mettant le plus à l'épreuve un groupe ; il faut clairement des rôles définis pour éviter de se marcher les uns sur les autres. Au sein du groupe, cela s'est bien déroulé avec un attelage clair ; personnellement, je me suis occupé de la page d'accueil et du style global des pages, en finissant par faire la "fusion" entre toutes les pages (les relier). 

\subsection{En images}

\subsection{Qu'ai-je appris ?}

Le développement Web, surtout en front-end, est une de mes plus grandes faiblesses en terme d'informatique. Néanmoins, j'ai essayé au mieux de surpasser cela en m'y investissant beaucoup, car cette note comptait pour deux SAé. Le groupe a travaillé bien ensemble pour développer le site Web, malgré les défis posés par la concurrence et la nécessité de clairement définir les rôles. Personnellement, j'ai appris à gérer mes faiblesses en front-end et à surmonter les obstacles pour acheter le développement du site.

Ainsi j'ai pu travailler sur l'implémentation d'une solution en équipe en apprenant à bien répartir les tâches afin de tirer le maximum des qualités et capacités de développement de chacun.

\tsecnonum{Compétences techniques}
\begin{itemize}
	\item Développement web en équipe
	\item Création de formulaires d'inscription
	\item Gestion du style global d'un site web
\end{itemize}

\tsecnonum{Savoir-être}
\begin{itemize}
	\item Travail en équipe et répartition des tâches
	\item Capacité à tirer parti des qualités de chacun
\end{itemize}

\section{SAÉ 2.01-02 - Logiciel de gestion d'échanges en Java}

\subsection{Présentation}

Cette SAé est la plus lourde de toute l'année, mais elle aussi la plus professionnelle.
Il est attendu de nous des rendus réguliers, sous forme de commits taggés.

Cette SAé intervient dans trois matières :

\begin{itemize}
	\item Graphes
	\item Programmation Orientée Objet (POO)
	\item Interaction Humain Machine (IHM)
\end{itemize}

Ces trois parties se déroulent dans leur ordre respectif, afin de converger à la fin vers un logiciel rassemblant toutes les fonctionnalités.

Ce logiciel consiste à organiser des échanges culturels entre pays et à déterminer les meilleurs échanges selon un algorithme de recherche d'appariement. Cet algorithme affecte un score à un échange en fonction de l'affinité des deux personnes (déterminée par leurs points communs et critères satisfaits). Le meilleur appariement est la combinaison d'échanges uniques pour laquelle la somme des affinités est minimale.

\subsection{Quelle était l'application ?}

L'utilité claire est de s'organiser autour d'un lourd projet en organisant plusieurs parties. L'application consiste à délivrer un logiciel fonctionnel répondant à des demandes utilisateur, en nous mettant alors dans la peau d'une vraie équipe de développeurs ayant des critères à satisfaire. Ce logiciel devait, par exemple, calculer les meilleurs échanges, permettre de bannir un échange, ou encore en rendre un obligatoire (demandes du cahier des charges). Ce logiciel a été développé dans le langage Java pour la partie backend, puis en JavaFX pour la partie frontend.

\subsection{En images}

\insererfigure{img/201/1.png}{10cm}{Page d'accueil}{fig::201::1}

\insererfigure{img/201/2.png}{8cm}{Liste des échanges}{fig::201::2}

\insererfigure{img/201/3.png}{5cm}{Clic sur un échange}{fig::201::3}

\subsection{Qu'ai-je appris ?}

\section{SAÉ 2.03 - Installation de services réseaux}

\subsection{Présentation}

\subsection{Quelle était l'application ?}

\subsection{Qu'ai-je appris ?}

\section{SAÉ 2.05 - Gestion d'un projet}

\subsection{Présentation}

Nous allons maintenant dans la ressource "Gestion" ! Cette matière nous donne à voir la gestion des entreprises en relation avec l'informatique. Dans ce projet, se déroulant en équipe de 5 pour notre part, nous avons créé une application mobile fictive en présentant son environnement économique, sa cible type, ses concurrents... Il s'agit globalement d'une étude sur la viabilité de notre application et notre stratégie.

\subsection{Quelle était l'application ?}

Nous avons choisi de créer Fridgerator, une application mobile de gestion de cuisine proposant la nouveauté hors pair de pouvoir d'une simple photo de son frigo générer des recettes viables grâce à l'IA. Ces applications sont typiques en 2025 et c'est pour quoi nous avons eu à coeur de nous renseigner sur cela.

\subsection{En images}

\insererfigure{img/205/1.png}{10cm}{}{fig::}

\insererfigure{img/205/2.png}{10cm}{}{fig::}

\insererfigure{img/205/3.png}{10cm}{}{fig::}

\subsection{Qu'ai-je appris ?}

Ce projet m’a permis de comprendre que créer une application ne commence pas par le code, mais par une idée forte, un besoin bien identifié, et une organisation structurée. J’ai pu découvrir plusieurs facettes du rôle d’un entrepreneur numérique, en sortant du cadre strictement technique. Qu'est-ce que j'ai appris ? J'ai appris que créer une application mobile n'est pas simplement un défi technique, mais également un processus de conception et de développement qui nécessite une compréhension profonde des besoins du marché et des utilisateurs. J'ai pu mettre en évidence les étapes clés d'un projet numérique, y compris la définition de l'idée, la recherche de données, le développement, le testing et la mise en production. En outre, j'ai appris à travailler en équipe avec des collègues passionnés et compétents, ce qui a permis une meilleure coordination et une amélioration de la qualité finale de l'application.

\tsecnonum{Compétences techniques}

\begin{itemize}
	\item Réalisation d’une \textbf{étude de marché} à partir de sources fiables pour identifier les opportunités et les menaces du secteur.
	\item Élaboration d’un \textbf{business model} clair et structuré (cible, valeur, canaux, revenus).
	\item Utilisation d’outils collaboratifs pour \textbf{planifier un projet en équipe} (tableaux Trello, Google Docs, etc.).
	\item Construction d’un \textbf{argumentaire marketing} basé sur les besoins utilisateurs.
	\item Notions de \textbf{positionnement stratégique} et différenciation concurrentielle.
\end{itemize}

\tsecnonum{Savoir-être}

\begin{itemize}
	\item Capacité à \textbf{travailler en équipe}, en répartissant efficacement les rôles selon les compétences de chacun.
	\item \textbf{Autonomie} dans la recherche d’informations pertinentes et l’organisation du travail.
	\item \textbf{Esprit critique} dans l’analyse de la faisabilité et la pertinence des idées.
	\item \textbf{Créativité} pour imaginer une solution innovante répondant à un besoin réel.
	\item \textbf{Communication claire et synthétique}, essentielle pour défendre une idée et convaincre un public.
\end{itemize}

\section{SAÉ 2.06 - Réalisation d'une vidéo sur un point de science informatique}

\subsection{Présentation}

\subsection{Quelle était l'application ?}

\subsection{Qu'ai-je appris ?}

2 pages minimum par section.

clarté 
concision
compréhension
conviction

\end{document}

